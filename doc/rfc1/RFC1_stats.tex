\documentclass{article}
\usepackage[utf8]{inputenc}

\title{RFC stats}
\author{Chloe Yugawa}
\date{November 2017}

\begin{document}

\maketitle

\section{Introduction}
Signals that collide are problematic with the current proposed solution to this problem. In this section, we will explore the probability that signals will collide in a few different cases. Assumptions for this section are as follows:
\begin{itemize}
\item The number of crabs transmitting in the entire study is 500
\item The target listening radius is 10 meters with an area of $314m^2$
\item The estimated density of tagged crabs is 0.00012 tagged crabs/m2 (that's 1 tagged crab per 8500 square meters)
\item The conservative estimate of tagged crabs is 0.001 crabs/m2 (1 tagged crab per 1000 square meters).  The average distance between crabs will be about 32 m if they're all evenly spaced.
\item For the sake of simplicity, a signal is defined as the total time between the beginning of a signal and the end of the signal, including the encoded silence. The time for this is conservatively estimated to be 2 seconds
\item The signal space is defined as the time between the end of one signal and the start of the next. This will be between 0 and 5 seconds.
\item The total signal time will then be between 2 and 7 seconds (signal + signal space). On average, that's 20 signals per minute
\item A conservative estimate for the percentage of crabs tagged is $.5\% $
\item The probability of finding more than one crab within a meter (tagged or untagged) during the season when research will occur is $3-5\%$

\end{itemize}


\newline \newline

First, we use a Poisson distribution. For one minute, a signal crab will produce an average of 20 signals, and $\lambda = 3$. For a period of 5 seconds, the probability of one signal occurring is $P(s)=\int_{0}^{.2}3*e^{-3t}dt=0.45$. Assuming a uniform distribution, the probability of any two crabs signaling in the same 5 second time frame is $P(two\_signals)=P(s)*P(s) = 0.2025$. \newline 

Next, we explore how likely finding two crabs within the listening radius (10 meters). For this section, we will use the conservative estimate of one tagged crab in $1000m^2$. The probability of finding one tagged crab within the listening area of $314m^2$ is $P(crab)=0.314$. The probability of finding two tagged crabs in that same area is $P(two\_crabs)=0.099$. The probability that there are two tagged crabs withing the listening radius that transmit at the same time is $P(two\_signals)*P(two\_crabs)=0.02$. 

%%Add something about the whole 3-5\% thing? In relation to population and percentage of tagged crabs? 


\end{document}
