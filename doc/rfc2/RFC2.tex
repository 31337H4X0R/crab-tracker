\documentclass[12pt]{article}

\setlength{\parindent}{0pt}
\setlength{\parskip}{2mm}

\usepackage{geometry}
 \geometry{
 letterpaper, left=20mm, right=20mm,  top=20mm,
 }
\usepackage{graphicx}
\graphicspath{ {graphics/} }
\usepackage{amssymb}

%%%%%%%%%%%%%%%%%%%%%%%%%%%%%%%%%%%%%%%%%%%%%%%%%
\title{RFC 2 - Boolean Value Encoding in Transmission Protocol}
%%%%%%%%%%%%%%%%%%%%%%%%%%%%%%%%%%%%%%%%%%%%%%%%%

\author{
	Noah Strong
}

\date{\today\ -- v1.0-wip}

\begin{document}

\maketitle

\tableofcontents{}

\section{Introduction}

We would like to find a way to encode a binary value in each transmission
along with the UID.
The encoding must not interfere with our current collision detection work
and should not require a massive overhall of the transmission protocol as it
currently stands.

\section{Background}

Patrick has requested that we look into adding the ability for a transmitter to
broadcast some additional detail about the crab it is paired with.
Specifically, Patrick is interested in attaching accelerometers to the tagged
crabs so that we can determine if a given crab is ``inert,'' perhaps because it
has molted its shell or because it has died. We would encode this as an
additional binary value in the transmission signal.

Patrick has stated that this is a highly desirable feature, but not absolutely
critical to the end product.

\section{Requirements and Concerns}

We want to be able to encode both a UID and some binary value in each
transmission without losing the work we've done to ensure that all collisions
are detectable. Additionally, in the event that we decide not to include this
feature in the final product, the changes made to the iCRAB protocol must not
interfere with our original goals. That is, if we scrap this requirement, it
should cost little or no extra work to continue on our original path.

We have already created a protocol that will encode a UID and is robust against
collisions (that is, they are detectable).
Some methods for adding binary encoding may interfere with the this collision
detection work, and we would prefer that such solutions be avoided.
In other words, collisions should still be detectable, at least in most cases,
even if addition information is encoded in the signal.

\section{Potential Solutions}

In the case that the given boolean value that we wish to express is
{\bf false}, we may broadcast the original signal with no adjustments.
We next propose alterations we could make in the case that the given boolean
value we with to express is {\bf true}.

\subsection{Add Additional Ping}

In this solution, a single UTP would be represented as a sequence such as
{\em P\_P\_P} where $P$ is a ping and $\_$ is a delay.

While this would work in the general case, it adds some additional complexity
to the receiver's detection code.
Worse, it may cause collisions to be undetectable under certain circumstances.
We will not discuss these situations, but they are easily discoverable and
therefore left as an exercise to the interested reader.

\subsection{Adjust Duration of Both Pings}

One variation of this is essentially double the number of IDs we can encode.
Half of the IDs are unchanged.
The other half of IDs (eg the larger IDs, or the even-numbered IDs, or
something along those lines) are reserved for transmitters to use when their
boolean value has become TRUE.

For example, suppose we have 500 unique IDs, but choose to only assign the
even-numbered IDs.
The odd-numbered IDs could then be used for existing transmitters when their
boolean value $B$ is TRUE.
Then transmitter 42, for example, would transmit the number 42 while $b$ was
false.
Once $b$ becomes true, that transmitter will start transmitting the value 43.
Since 43 is an odd number, the receiver could easily deduce that transmitter
42's $b$ value was TRUE.

Alternatively, suppose we have, for example, 500 IDs, and we assign 0-499 to
transmitters.
We then use values in the range 500-999 only when $b$ is TRUE.
That is, transmitter 42 would transmit 42 when $b=$ FALSE and 500+42=542 when
$b=$ TRUE.

This method requires very little adjustment to the encoding protocol.
Instead, it would require only a small adjustment to how we interpret ID values
as they are detected.

However, this method does double the number of IDs we need to be able to
encode, which means that the max UTP duration will be much higher.
This could potentially lead to more collisions.
This solution is also potentially confusing, as the encoding is not obvious
and it may rely on magic numbers.

\subsection{Adjust Duration of Only One Ping}

\end{document}

